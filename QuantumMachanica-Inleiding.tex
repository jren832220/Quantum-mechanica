\documentclass{report} % 'report' ondersteunt wel standaard de abstract
\usepackage[utf8]{inputenc}
\usepackage[T1]{fontenc}
\usepackage[dutch]{babel}
\usepackage{amsmath, amssymb}
\usepackage{tikz}
\usepackage{pgfplots}
\usepackage{epigraph}
\usepackage{hyperref}

\title{Quantum Mechanica Inleiding}
\author{Jeroen Vermeulen}
\date{}

\begin{document}
	\maketitle
	
	\begin{abstract}
		Dit document biedt een kritische inleiding tot de overgang van de klassieke natuurkunde naar de moderne kwantummechanica, met een specifieke focus op de evolutie van atoommodellen. Centraal staat het atoommodel van Bohr, dat een revolutionaire stap betekende door de introductie van kwantisering in de atomaire structuur. 
		
		De kern van deze verhandeling analyseert de drie postulaten van Bohr:
		\begin{enumerate}
			\item De existentie van stationaire banen waarin elektronen geen straling uitzenden.
			\item De voorwaarde voor kwantisering van het impulsmoment: $L = n\hbar$.
			\item De frequentievoorwaarde bij elektronenovergangen tussen energieniveaus.
		\end{enumerate}
		
		Hoewel het model van Bohr succesvol de spectraallijnen van waterstof verklaarde, worden de fundamentele beperkingen ervan besproken, zoals het onvermogen om spectra van complexere atomen te voorspellen en het negeren van de golf-deeltje dualiteit. Tot slot wordt uiteengezet hoe deze tekortkomingen de noodzaak creëerden voor het kwantummechanisch atoommodel van Schrödinger. Hierbij verschuift het paradigma van deterministische banen naar de probabilistische benadering van de golffunctie ($\psi$), gebaseerd op de fundamentele Schrödingervergelijking.
	\end{abstract}
	
	\chapter{Inleiding}
	...
	
\end{document}