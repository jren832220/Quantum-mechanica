\documentclass{report} % 'report' ondersteunt wel standaard de abstract
\usepackage[utf8]{inputenc}
\usepackage[T1]{fontenc}
\usepackage[dutch]{babel}
\usepackage{amsmath, amssymb}
\usepackage{tikz}
\usepackage{pgfplots}
\usepackage{epigraph}
\usepackage{hyperref}

\title{Quantum Mechanica Inleiding}
\author{Jeroen Vermeulen}
\date{}

\begin{document}
	\maketitle
	\tableofcontents
	\begin{abstract}
		Dit document biedt een kritische inleiding tot de overgang van de klassieke natuurkunde naar de moderne kwantummechanica, met een specifieke focus op de evolutie van atoommodellen. Centraal staat het atoommodel van Bohr, dat een revolutionaire stap betekende door de introductie van kwantisering in de atomaire structuur. 
		
		De kern van deze verhandeling analyseert de drie postulaten van Bohr:
		\begin{enumerate}
			\item De existentie van stationaire banen waarin elektronen geen straling uitzenden.
			\item De voorwaarde voor kwantisering van het impulsmoment: $L = n\hbar$.
			\item De frequentievoorwaarde bij elektronenovergangen tussen energieniveaus.
		\end{enumerate}
		
		Hoewel het model van Bohr succesvol de spectraallijnen van waterstof verklaarde, worden de fundamentele beperkingen ervan besproken, zoals het onvermogen om spectra van complexere atomen te voorspellen en het negeren van de golf-deeltje dualiteit. Tot slot wordt uiteengezet hoe deze tekortkomingen de noodzaak creëerden voor het kwantummechanisch atoommodel van Schrödinger. Hierbij verschuift het paradigma van deterministische banen naar de probabilistische benadering van de golffunctie ($\psi$), gebaseerd op de fundamentele Schrödingervergelijking.
	\end{abstract}
	
	\chapter{Het atoommodel van Bohr}
	\epigraph{“The more success the quantum theory has, the sillier it looks.”}{Albert Einstein}

	Het atoommodel van Bohr, geïntroduceerd in 1913, was een baanbrekende stap in de ontwikkeling van de kwantummechanica.
	Het gaf ons een eerste inzicht in de structuur van het atoom en de manier waarop elektronen zich gedragen.
	Bohr stelde drie postulaten voor die de structuur van het atoom beschrijven:
	\begin{itemize}
		\item Elektronen mogen alleen bewegen in bepaalde banen (schillen) rond de kern.
		Deze banen hebben een vaste energie en worden stationaire banen genoemd.
		Er is geen energieverlies door straling zolang het elektron zich in een stationaire baan bevindt.
		\item Kwantificatie van energie: \\
		Niet alle banen zijn toegestaan; alleen die banen waarbij het impulsmoment van het elektron een geheel veelvoud is van $\hbar$ (de gereduceerde Planck-constante) zijn toegestaan. Dit wordt uitgedrukt als $L = n\hbar$, waarbij $n$ een positief geheel getal is.\\
		De energie van een elektron in een stationaire baan is kwantumgequantificeerd en kan worden berekend met behulp van de formule $E_n = -\frac{13.6 \text{ eV}}{n^2}$ voor het waterstofatoom. $n$ is ook hier een geheel getal, ook wel het hoofdkwantumgetal genoemd.
		\item Lichtemissie en -absorptie: \\
		Een elektron kan van de ene stationaire baan naar de andere springen door het absorberen of uitzenden van een foton.
		De energie van het foton komt overeen met het verschil in energie tussen de twee banen, wat wordt uitgedrukt als $E_{\text{foton}} = E_{\text{hoger}} - E_{\text{lager}}$.
		Een elektron zendt licht uit wanneer het van een hogere naar een lagere baan springt, en absorbeert licht wanneer het van een lagere naar een hogere baan springt.
		De energie van het foton kan ook worden uitgedrukt in termen van de frequentie van het licht: $E_{\text{foton}} = h\nu$, waarbij $h$ de Planck-constante is en $\nu$ de frequentie van het licht.
		De Plank-constante is een fundamentele constante in de natuurkunde die de kwantisering van energie en impulsmoment beschrijft en heeft een waarde van ongeveer $6.626 \times 10^{-34}$ joule-seconden.
		De frequentie van het licht is gerelateerd aan de golflengte door de formule $\nu = \frac{c}{\lambda}$, waarbij $c$ de snelheid van het licht is en $\lambda$ de golflengte van het licht.
		Hoe hoger de energie van het foton, hoe hoger de frequentie en hoe korter de golflengte van het licht. Hoe lager de energie van het foton, hoe lager de frequentie en hoe langer de golflengte van het licht.
		Hoe hoger de frequentie van het licht, hoe meer energie het foton heeft. Hoe lager de frequentie van het licht, hoe minder energie het foton heeft. Zo hebben de kleuren rood en oranje een lagere frequentie en minder energie dan de kleuren blauw en violet.
	\end{itemize}
	\section{Waarom werkt het model van Bohr?}
	Het model van Bohr verklaart:
	\begin{itemize}
		\item De stabiliteit van atomen: \\
		Volgens de klassieke natuurkunde zouden elektronen continu energie moeten verliezen door straling terwijl ze rond de kern draaien, wat zou leiden tot instabiliteit van atomen.
		\item Het emissiespectrum van waterstof: \\
		Bohr's model verklaart de discrete lijnen in het emissiespectrum van waterstof, die overeenkomen met de energieverschillen tussen de stationaire banen.
		\item waarom energiëniveaus van atomen kwantumgequantificeerd zijn: \\
	\end{itemize}
   volgens het waterstofatoom klopt het exact met de experimenten.
	\section{Het Bohr-model heeft beperkingen}
	Hoewel het model van Bohr succesvol was in het verklaren van het gedrag van het waterstofatoom, heeft het verschillende beperkingen:
	\begin{itemize}
		\item Het model kan niet de spectra van complexere atomen verklaren, zoals helium of lithium, omdat het geen rekening houdt met de interacties tussen meerdere elektronen.
		\item Het model negeert de golf-deeltje dualiteit van elektronen, wat later werd erkend als een fundamenteel aspect van de kwantummechanica.
		\item Het model is niet in staat om de fijne structuur van spectraallijnen te verklaren, die wordt veroorzaakt door relativistische effecten en de spin van elektronen.
		\item elektronen bewegen niet echt in cirkelvormige banen, maar hebben een meer complexe beweging die beter wordt beschreven door de golffunctie in de Schrödingervergelijking.
	\end{itemize}
	Daarom zal het later ook vervangen worden door het kwantummechanisch atoommodel van Schrödinger, dat een meer nauwkeurige en uitgebreide beschrijving biedt van de structuur en het gedrag van atomen.
	\chapter{Het atoommodel van Bohr wiskundig gezien}
	Nu we de postulaten van Bohr hebben besproken, kunnen we deze wiskundig formuleren en analyseren. \\
	We beginnen met een herhaling van de zaken die we in het vorige hoofdstuk hebben besproken:
	\section{Postulaten (Bohr, 1913)}
	Bohr introduceerde (in moderne notatie) de volgende kernidee\"en voor het waterstofachtige atoom (kernlading $+Ze$ en \'e\'en elektron):
	\begin{enumerate}
		\item Het elektron kan zich enkel in \emph{stationaire banen} bevinden zonder straling uit te zenden.
		\item Enkel bij een overgang $n\to m$ wordt straling uitgezonden/geabsorbeerd met
		\[
			h\nu = E_n - E_m.
		\]
		\item De baanimpuls is gequantiseerd:
		\[
			L = m_e v r = n\hbar,\qquad n\in\mathbb{N}.
		\]
	\end{enumerate}

	\end{document}