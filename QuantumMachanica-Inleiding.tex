\documentclass{report} % 'report' ondersteunt wel standaard de abstract
\usepackage[utf8]{inputenc}
\usepackage[T1]{fontenc}
\usepackage[dutch]{babel}
\usepackage{amsmath, amssymb}
\usepackage{tikz}
\usepackage{pgfplots}
\usepackage{epigraph}
\usepackage{hyperref}

\title{Quantum Mechanica Inleiding}
\author{Jeroen Vermeulen}
\date{}

\begin{document}
	\maketitle
	\tableofcontents
	\begin{abstract}
		Dit document biedt een kritische inleiding tot de overgang van de klassieke natuurkunde naar de moderne kwantummechanica, met een specifieke focus op de evolutie van atoommodellen. Centraal staat het atoommodel van Bohr, dat een revolutionaire stap betekende door de introductie van kwantisering in de atomaire structuur. 
		
		De kern van deze verhandeling analyseert de drie postulaten van Bohr:
		\begin{enumerate}
			\item De existentie van stationaire banen waarin elektronen geen straling uitzenden.
			\item De voorwaarde voor kwantisering van het impulsmoment: $L = n\hbar$.
			\item De frequentievoorwaarde bij elektronenovergangen tussen energieniveaus.
		\end{enumerate}
		
		Hoewel het model van Bohr succesvol de spectraallijnen van waterstof verklaarde, worden de fundamentele beperkingen ervan besproken, zoals het onvermogen om spectra van complexere atomen te voorspellen en het negeren van de golf-deeltje dualiteit. Tot slot wordt uiteengezet hoe deze tekortkomingen de noodzaak creëerden voor het kwantummechanisch atoommodel van Schrödinger. Hierbij verschuift het paradigma van deterministische banen naar de probabilistische benadering van de golffunctie ($\psi$), gebaseerd op de fundamentele Schrödingervergelijking.
	\end{abstract}
	
	\chapter{Het atoommodel van Bohr}
	\epigraph{“The more success the quantum theory has, the sillier it looks.”}{Albert Einstein}

	Het atoommodel van Bohr, geïntroduceerd in 1913, was een baanbrekende stap in de ontwikkeling van de kwantummechanica.
	Het gaf ons een eerste inzicht in de structuur van het atoom en de manier waarop elektronen zich gedragen.
	Bohr stelde drie postulaten voor die de structuur van het atoom beschrijven:
	\begin{itemize}
		\item Elektronen mogen alleen bewegen in bepaalde banen (schillen) rond de kern.
		Deze banen hebben een vaste energie en worden stationaire banen genoemd.
		Er is geen energieverlies door straling zolang het elektron zich in een stationaire baan bevindt.
		\item Kwantificatie van energie: \\
		Niet alle banen zijn toegestaan; alleen die banen waarbij het impulsmoment van het elektron een geheel veelvoud is van $\hbar$ (de gereduceerde Planck-constante) zijn toegestaan. Dit wordt uitgedrukt als $L = n\hbar$, waarbij $n$ een positief geheel getal is.\\
		De energie van een elektron in een stationaire baan is kwantumgequantificeerd en kan worden berekend met behulp van de formule $E_n = -\frac{13.6 \text{ eV}}{n^2}$ voor het waterstofatoom. $n$ is ook hier een geheel getal, ook wel het hoofdkwantumgetal genoemd.
		\item Lichtemissie en -absorptie: \\
		Een elektron kan van de ene stationaire baan naar de andere springen door het absorberen of uitzenden van een foton.
		De energie van het foton komt overeen met het verschil in energie tussen de twee banen, wat wordt uitgedrukt als $E_{\text{foton}} = E_{\text{hoger}} - E_{\text{lager}}$.
		Een elektron zendt licht uit wanneer het van een hogere naar een lagere baan springt, en absorbeert licht wanneer het van een lagere naar een hogere baan springt.
		De energie van het foton kan ook worden uitgedrukt in termen van de frequentie van het licht: $E_{\text{foton}} = h\nu$, waarbij $h$ de Planck-constante is en $\nu$ de frequentie van het licht.
		De Plank-constante is een fundamentele constante in de natuurkunde die de kwantisering van energie en impulsmoment beschrijft en heeft een waarde van ongeveer $6.626 \times 10^{-34}$ joule-seconden.
		De frequentie van het licht is gerelateerd aan de golflengte door de formule $\nu = \frac{c}{\lambda}$, waarbij $c$ de snelheid van het licht is en $\lambda$ de golflengte van het licht.
		Hoe hoger de energie van het foton, hoe hoger de frequentie en hoe korter de golflengte van het licht. Hoe lager de energie van het foton, hoe lager de frequentie en hoe langer de golflengte van het licht.
		Hoe hoger de frequentie van het licht, hoe meer energie het foton heeft. Hoe lager de frequentie van het licht, hoe minder energie het foton heeft. Zo hebben de kleuren rood en oranje een lagere frequentie en minder energie dan de kleuren blauw en violet.
	\end{itemize}
	\section{Waarom werkt het model van Bohr?}
	Het model van Bohr verklaart:
	\begin{itemize}
		\item De stabiliteit van atomen: \\
		Volgens de klassieke natuurkunde zouden elektronen continu energie moeten verliezen door straling terwijl ze rond de kern draaien, wat zou leiden tot instabiliteit van atomen.
		\item Het emissiespectrum van waterstof: \\
		Bohr's model verklaart de discrete lijnen in het emissiespectrum van waterstof, die overeenkomen met de energieverschillen tussen de stationaire banen.
		\item waarom energiëniveaus van atomen kwantumgequantificeerd zijn: \\
	\end{itemize}
   volgens het waterstofatoom klopt het exact met de experimenten.
	\section{Het Bohr-model heeft beperkingen}
	Hoewel het model van Bohr succesvol was in het verklaren van het gedrag van het waterstofatoom, heeft het verschillende beperkingen:
	\begin{itemize}
		\item Het model kan niet de spectra van complexere atomen verklaren, zoals helium of lithium, omdat het geen rekening houdt met de interacties tussen meerdere elektronen.
		\item Het model negeert de golf-deeltje dualiteit van elektronen, wat later werd erkend als een fundamenteel aspect van de kwantummechanica.
		\item Het model is niet in staat om de fijne structuur van spectraallijnen te verklaren, die wordt veroorzaakt door relativistische effecten en de spin van elektronen.
		\item elektronen bewegen niet echt in cirkelvormige banen, maar hebben een meer complexe beweging die beter wordt beschreven door de golffunctie in de Schrödingervergelijking.
	\end{itemize}
	Daarom zal het later ook vervangen worden door het kwantummechanisch atoommodel van Schrödinger, dat een meer nauwkeurige en uitgebreide beschrijving biedt van de structuur en het gedrag van atomen.
	\chapter{Het atoommodel van Bohr wiskundig gezien}
	Nu we de postulaten van Bohr hebben besproken, kunnen we deze wiskundig formuleren en analyseren. \\
	We beginnen met een herhaling van de zaken die we in het vorige hoofdstuk hebben besproken:
	\section{Postulaten (Bohr, 1913)}
	Bohr introduceerde (in moderne notatie) de volgende kernidee\"en voor het waterstofachtige atoom (kernlading $+Ze$ en \'e\'en elektron):
	\begin{enumerate}
		\item Het elektron kan zich enkel in \emph{stationaire banen} bevinden zonder straling uit te zenden.
		\item Enkel bij een overgang $n\to m$ wordt straling uitgezonden/geabsorbeerd met
		\[
			h\nu = E_n - E_m.
		\]
		\item De baanimpuls is gequantiseerd:
		\[
			L = m_e v r = n\hbar,\qquad n\in\mathbb{N}.
		\]
	\end{enumerate}
	\section{Dynamica: Coulombkracht als centripetale kracht}
	Neem een cirkelbaan met straal $r$ en snelheid $v$. De Coulombkracht levert de centripetale kracht:
	\begin{align}
		\frac{m_e v^2}{r} &= \frac{1}{4\pi\varepsilon_0}\frac{Z e^2}{r^2}.
		\label{eq:coulomb_centripetaal}
	\end{align}
	De uitkomst wordt uitgedrukt in Newton (N)\\

	Definieer voor compactheid
	\[
		k \equiv \frac{1}{4\pi\varepsilon_0}.
	\]
	Dan wordt \eqref{eq:coulomb_centripetaal}
	\begin{align}
		m_e v^2 &= \frac{k Z e^2}{r}.
		\label{eq:mv2}
	\end{align}

	e is de elementaire lading en wordt uitgedrukt in Coulomb (C)\\
	$\varepsilon_0$ is de permittiviteit van het vacuüm en wordt uitgedrukt in Farad per meter (F/m)\\
	r is de afstand van het elektron tot de kern en wordt uitgedrukt in meter (m)\\
	$m_e$ is de massa van het elektron en wordt uitgedrukt in kilogram (kg)\\
	v is de snelheid van het elektron en wordt uitgedrukt in meter per seconde (m/s)\\
	\section{Quantisatie van het impulsmoment en straal van de banen}
	Uit $L=m_e v r = n\hbar$ volgt
	\begin{align}
		v &= \frac{n\hbar}{m_e r}.
		\label{eq:v_from_L}
	\end{align}
	Combineer \eqref{eq:mv2} en \eqref{eq:v_from_L}:
	\begin{align}
		m_e\left(\frac{n\hbar}{m_e r}\right)^2 &= \frac{k Z e^2}{r}
		\\
		\frac{n^2\hbar^2}{m_e r^2} &= \frac{k Z e^2}{r}
		\\
		r_n &= \frac{n^2\hbar^2}{m_e k Z e^2}.
		\label{eq:bohr_radius_general}
	\end{align}
	Voor $Z=1$ definieert men de \emph{Bohrstraal}
	\begin{align}
		a_0 \equiv \frac{\hbar^2}{m_e k e^2} = \frac{4\pi\varepsilon_0\hbar^2}{m_e e^2},
	\end{align}
	zodat
	\begin{align}
		r_n = \frac{n^2}{Z}\,a_0.
	\end{align}
	waarbij $\hbar$ de gereduceerde Planck-constante is,
	$m_e$ de massa van het elektron,
	$k$ de constante gedefinieerd als $\frac{1}{4\pi\varepsilon_0}$, $Z$ de kernlading,
	en $e$ de elementaire lading. De Bohrstraal
	$a_0$ is een fundamentele lengte-eenheid in de atoomfysica en heeft een waarde van ongeveer 0.529 Ångström (Å), wat overeenkomt met de straal van de grondtoestand van het waterstofatoom.
	\section{Snelheden en de fijnstructuurconstante}
	Uit \eqref{eq:mv2} en \eqref{eq:bohr_radius_general} volgt ook een eenvoudige uitdrukking voor $v_n$.
	Gebruik \eqref{eq:v_from_L} met $r=r_n$:
	\begin{align}
		v_n &= \frac{n\hbar}{m_e r_n}
		= \frac{n\hbar}{m_e}\,\frac{m_e k Z e^2}{n^2\hbar^2}
		= \frac{k Z e^2}{n\hbar}.
	\end{align}
	Met $\alpha$ de fijnstructuurconstante,
	\[
		\alpha \equiv \frac{e^2}{4\pi\varepsilon_0\hbar c} = \frac{k e^2}{\hbar c},
	\]
	krijg je
	\begin{align}
		\frac{v_n}{c} = \frac{Z\alpha}{n}.
	\end{align}
	Dit toont meteen de \emph{consistentievoorwaarde} voor een niet-relativistische behandeling: $Z\alpha/n\ll 1$.
\\
	waarbij:\\
	$v_n$ de snelheid van het elektron in baan $n$ is en wordt uitgedrukt in meter per seconde (m/s)\\
	$n$ het hoofdkwantumgetal is (een positief geheel getal)\\
	$\hbar$ de gereduceerde Planck-constante is ($\hbar = h/2\pi \approx 1.055 \times 10^{-34}$ J·s)\\
	$m_e$ de massa van het elektron is ($\approx 9.109 \times 10^{-31}$ kg)\\
	$r_n$ de straal van de $n$-de baan is en wordt uitgedrukt in meter (m)\\
	$k$ de constante is gedefinieerd als $\frac{1}{4\pi\varepsilon_0}$ ($\approx 8.988 \times 10^9$ N·m²/C²)\\
	$Z$ de kernlading is (het aantal protonen in de kern)\\
	$e$ de elementaire lading is ($\approx 1.602 \times 10^{-19}$ C)\\
	$\alpha$ de fijnstructuurconstante is, een dimensieloze natuurconstante ($\alpha \approx 1/137 \approx 0.0073$)\\
	$\varepsilon_0$ de permittiviteit van het vacuüm is ($\approx 8.854 \times 10^{-12}$ F/m)\\
	$c$ de lichtsnelheid in vacuüm is ($\approx 2.998 \times 10^8$ m/s)\\

	\section{Energieën van de stationaire toestanden}
	De totale mechanische energie is
	\[
		E = T + V,
	\]
	met kinetische energie $T=\frac12 m_e v^2$ en potenti\"ele energie
	\[
		V(r) = -k\frac{Z e^2}{r}.
	\]
	Gebruik \eqref{eq:mv2}: $m_e v^2 = \frac{k Z e^2}{r}$, dus
	\begin{align}
		T &= \frac12\,\frac{k Z e^2}{r},
		\\
		E &= \frac12\,\frac{k Z e^2}{r} - \frac{k Z e^2}{r}
		= -\frac12\,\frac{k Z e^2}{r}.
		\label{eq:E_vs_r}
	\end{align}
	Invullen van $r=r_n$ uit \eqref{eq:bohr_radius_general} geeft
	\begin{align}
		E_n
		&= -\frac12\,\frac{k Z e^2}{r_n}
		= -\frac12\,k Z e^2\left(\frac{m_e k Z e^2}{n^2\hbar^2}\right)
		\\
		&= -\frac{m_e k^2 Z^2 e^4}{2\hbar^2}\,\frac{1}{n^2}.
		\label{eq:bohr_energy_general}
	\end{align}
	Voor waterstof ($Z=1$) is
	\[
		E_n = -\frac{m_e k^2 e^4}{2\hbar^2}\frac{1}{n^2}
		\approx -\frac{13.6\,\mathrm{eV}}{n^2}.
	\]
	\\
	voor de volledigheid geven we hier ook de betekenis van de symbolen die in deze formules voorkomen:
	waarbij:\\
	$E$ de totale mechanische energie is en wordt uitgedrukt in joule (J) of elektronvolt (eV)\\
	$T$ de kinetische energie is (bewegingsenergie van het elektron) en wordt uitgedrukt in joule (J)\\
	$V$ de potentiële energie is (elektrische energie door Coulombkracht) en wordt uitgedrukt in joule (J)\\
	$m_e$ de massa van het elektron is ($\approx 9.109 \times 10^{-31}$ kg)\\
	$v$ de snelheid van het elektron is en wordt uitgedrukt in meter per seconde (m/s)\\
	$k$ de constante is gedefinieerd als $\frac{1}{4\pi\varepsilon_0}$ ($\approx 8.988 \times 10^9$ N·m²/C²)\\
	$Z$ de kernlading is (het aantal protonen in de kern)\\
	$e$ de elementaire lading is ($\approx 1.602 \times 10^{-19}$ C)\\
	$r$ de afstand van het elektron tot de kern is en wordt uitgedrukt in meter (m)\\
	$E_n$ de energie van het elektron in de $n$-de stationaire baan is\\
	$r_n$ de straal van de $n$-de baan is en wordt uitgedrukt in meter (m)\\
	$n$ het hoofdkwantumgetal is (een positief geheel getal: $n = 1, 2, 3, \ldots$)\\
	$\hbar$ de gereduceerde Planck-constante is ($\hbar = h/2\pi \approx 1.055 \times 10^{-34}$ J·s)\\
	eV elektronvolt is, een energie-eenheid (1 eV $\approx 1.602 \times 10^{-19}$ J)\\
	\\
	De negatieve waarde van $E_n$ geeft aan dat het elektron gebonden is aan de kern. Hoe negatiever de energie, hoe sterker het elektron gebonden is. Bij $E = 0$ is het elektron vrij (geïoniseerd).

	\subsection{Spectrale lijnen: Rydberg-formule}
	Bij een overgang $n\to m$ ($n>m$) geldt Bohrs frequentievoorwaarde
	\[
		h\nu = E_n - E_m.
	\]
	Met \eqref{eq:bohr_energy_general}:
	\begin{align}
		h\nu
		&= \frac{m_e k^2 Z^2 e^4}{2\hbar^2}\left(\frac{1}{m^2}-\frac{1}{n^2}\right).
	\end{align}
	Gebruik $\nu=c/\lambda$ en $h=2\pi\hbar$:
	\begin{align}
		\frac{1}{\lambda}
		&= \frac{\nu}{c}
		= \frac{m_e k^2 Z^2 e^4}{2\hbar^2}\,\frac{1}{h c}\left(\frac{1}{m^2}-\frac{1}{n^2}\right)
		\\
		&= \left[\frac{m_e k^2 e^4}{4\pi\hbar^3 c}\right] Z^2
		\left(\frac{1}{m^2}-\frac{1}{n^2}\right).
	\end{align}
	De factor tussen vierkante haken definieert de \emph{Rydberg-constante} (in deze eenvoudige versie met $m_e$):
	\begin{align}
		R_\infty \equiv \frac{m_e k^2 e^4}{4\pi\hbar^3 c}
		= \frac{m_e e^4}{8\varepsilon_0^2 h^3 c}.
	\end{align}
	Dan volgt de klassieke Rydberg-formule:
	\begin{align}
		\frac{1}{\lambda} = R_\infty Z^2\left(\frac{1}{m^2}-\frac{1}{n^2}\right).
	\end{align}
	\\
	waarbij:\\
	$h$ de Planck-constante is ($h \approx 6.626 \times 10^{-34}$ J·s)\\
	$\nu$ de frequentie van het uitgezonden of geabsorbeerde licht is en wordt uitgedrukt in hertz (Hz of s$^{-1}$)\\
	$E_n$ de energie van het elektron in de $n$-de stationaire baan is\\
	$E_m$ de energie van het elektron in de $m$-de stationaire baan is\\
	$n$ het hogere energieniveau is (positief geheel getal, $n > m$)\\
	$m$ het lagere energieniveau is (positief geheel getal)\\
	$m_e$ de massa van het elektron is ($\approx 9.109 \times 10^{-31}$ kg)\\
	$k$ de constante is gedefinieerd als $\frac{1}{4\pi\varepsilon_0}$ ($\approx 8.988 \times 10^9$ N·m²/C²)\\
	$Z$ de kernlading is (het aantal protonen in de kern)\\
	$e$ de elementaire lading is ($\approx 1.602 \times 10^{-19}$ C)\\
	$\hbar$ de gereduceerde Planck-constante is ($\hbar = h/2\pi \approx 1.055 \times 10^{-34}$ J·s)\\
	$c$ de lichtsnelheid in vacuüm is ($\approx 2.998 \times 10^8$ m/s)\\
	$\lambda$ de golflengte van het uitgezonden of geabsorbeerde licht is en wordt uitgedrukt in meter (m)\\
	$R_\infty$ de Rydberg-constante is ($R_\infty \approx 1.097 \times 10^7$ m$^{-1}$)\\
	$\varepsilon_0$ de permittiviteit van het vacuüm is ($\approx 8.854 \times 10^{-12}$ F/m)\\
	\\
	De Rydberg-formule voorspelt de golflengtes van de spectrale lijnen die een waterstofachtig atoom uitzendt. Voor waterstof ($Z=1$) verklaart deze formule de bekende spectraalseries zoals de Lyman-serie ($m=1$), Balmer-serie ($m=2$), en Paschen-serie ($m=3$).
	\section{Verbetering: gereduceerde massa (kern niet oneindig zwaar)}
	Strikt genomen draait niet alleen het elektron rond de kern: beide draaien rond het massamiddelpunt.
	Vervang daarom $m_e$ door de \emph{gereduceerde massa}
	\[
		\mu = \frac{m_e M}{m_e+M},
	\]
	waar $M$ de kernmassa is. Dan worden
	\begin{align}
		r_n &= \frac{n^2\hbar^2}{\mu k Z e^2},
		\\
		E_n &= -\frac{\mu k^2 Z^2 e^4}{2\hbar^2}\frac{1}{n^2},
		\\
		R &= R_\infty\frac{\mu}{m_e}.
	\end{align}
	Dit verklaart een deel van het isotopen-effect in spectra.

	\section{De Broglie-herinterpretatie (brug naar golven)}
	Bohrs quantisatievoorwaarde $m_e v r = n\hbar$ kan ook worden gezien als een staande-golfvoorwaarde.
	De Broglie geeft de golflengte
	\[
		\lambda = \frac{h}{p}=\frac{h}{m_e v}.
	\]
	Een staande golf op een cirkel vereist
	\[
		2\pi r = n\lambda = n\frac{h}{m_e v}
		\quad\Longrightarrow\quad
		m_e v r = n\hbar,
	\]
	dus Bohrs postulaten kunnen (deels) gemotiveerd worden vanuit golfmechanica.\\
	Zoals we in het vorige hoofdstuk besproken hebben, heeft het model van Bohr enkele beperkingen.
	We sommen deze hier nog eens op met een diepere wiskundige uitleg:
	\section{Beperkingen (conceptueel belangrijk)}
	Hoewel Bohr het waterstofspectrum goed benadert, is het model fundamenteel semi-klassiek:
	\begin{itemize}
		\item Het veronderstelt \emph{klassieke} banen (in tegenspraak met moderne QM waar toestanden golffuncties zijn).
		\item Het verklaart niet systematisch fijnstructuur, Zeeman/Stark-effecten, intensiteiten en selectie-regels
		(daarvoor heb je Schr\"odinger/Dirac + perturbatietheorie nodig).
		\item Voor grote $Z$ worden relativistische en QED-correcties belangrijk omdat $v/c\sim Z\alpha$.
	\end{itemize}

	\subsection{Wat is QED?}
	QED staat voor \emph{Quantum Electrodynamics} (Kwantumelektrodynamica in het Nederlands) en is de kwantumtheorie van elektromagnetische interacties.
	Het is een van de meest nauwkeurige en succesvol geteste theorieën in de natuurkunde.

	Belangrijke aspecten van QED:
	\begin{itemize}
		\item \textbf{Fundamentele theorie:} QED beschrijft hoe geladen deeltjes (zoals elektronen) wisselwerken met fotonen (de kwanta van het elektromagnetische veld). Het combineert kwantummechanica met speciale relativiteitstheorie.

		\item \textbf{Ontwikkeling:} De theorie werd ontwikkeld in de late jaren 1940 door Richard Feynman, Julian Schwinger en Sin-Itiro Tomonaga, waarvoor zij in 1965 de Nobelprijs kregen.

		\item \textbf{QED-correcties in atoomfysica:} Voor het waterstofatoom voorspelt QED extreem kleine maar meetbare correcties op de energieniveaus, zoals:
		\begin{itemize}
			\item De \emph{Lamb-verschuiving}: een klein verschil tussen de $2S_{1/2}$ en $2P_{1/2}$ energieniveaus, veroorzaakt door kwantumfluctuaties van het elektromagnetische vacuüm.
			\item Correcties door \emph{vacuümpolarisatie}: virtuele elektron-positronparen beïnvloeden de effectieve lading van de kern.
			\item De afwijking van het magnetische moment van het elektron van de Dirac-waarde (het zogeheten \emph{anomale magnetische moment}, $g-2$).
		\end{itemize}

		\item \textbf{Nauwkeurigheid:} QED-berekeningen komen overeen met experimenten tot ongeveer 12 significante cijfers, wat het een van de meest nauwkeurig geteste theorieën in de wetenschap maakt.

		\item \textbf{Belang voor grote $Z$:} Voor atomen met hoge kernlading $Z$ worden QED-effecten belangrijker omdat de elektronsnelheid $v/c \sim Z\alpha$ toeneemt. Bij zeer zware elementen kunnen deze correcties niet meer verwaarloosd worden.
	\end{itemize}

	In het Bohr-model worden deze subtiele kwantumeffecten volledig genegeerd, wat verklaart waarom het model slechts een benadering is van de werkelijkheid.
\end{document}